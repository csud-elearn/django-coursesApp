% Generated by Sphinx.
\def\sphinxdocclass{report}
\documentclass[letterpaper,10pt,english]{sphinxmanual}
\usepackage[utf8]{inputenc}
\DeclareUnicodeCharacter{00A0}{\nobreakspace}
\usepackage{cmap}
\usepackage[T1]{fontenc}
\usepackage{babel}
\usepackage{times}
\usepackage[Bjarne]{fncychap}
\usepackage{longtable}
\usepackage{sphinx}
\usepackage{multirow}


\title{TM2014 Documentation}
\date{December 29, 2014}
\release{0.1}
\author{Keran Kocher}
\newcommand{\sphinxlogo}{}
\renewcommand{\releasename}{Release}
\makeindex

\makeatletter
\def\PYG@reset{\let\PYG@it=\relax \let\PYG@bf=\relax%
    \let\PYG@ul=\relax \let\PYG@tc=\relax%
    \let\PYG@bc=\relax \let\PYG@ff=\relax}
\def\PYG@tok#1{\csname PYG@tok@#1\endcsname}
\def\PYG@toks#1+{\ifx\relax#1\empty\else%
    \PYG@tok{#1}\expandafter\PYG@toks\fi}
\def\PYG@do#1{\PYG@bc{\PYG@tc{\PYG@ul{%
    \PYG@it{\PYG@bf{\PYG@ff{#1}}}}}}}
\def\PYG#1#2{\PYG@reset\PYG@toks#1+\relax+\PYG@do{#2}}

\expandafter\def\csname PYG@tok@gd\endcsname{\def\PYG@tc##1{\textcolor[rgb]{0.63,0.00,0.00}{##1}}}
\expandafter\def\csname PYG@tok@gu\endcsname{\let\PYG@bf=\textbf\def\PYG@tc##1{\textcolor[rgb]{0.50,0.00,0.50}{##1}}}
\expandafter\def\csname PYG@tok@gt\endcsname{\def\PYG@tc##1{\textcolor[rgb]{0.00,0.27,0.87}{##1}}}
\expandafter\def\csname PYG@tok@gs\endcsname{\let\PYG@bf=\textbf}
\expandafter\def\csname PYG@tok@gr\endcsname{\def\PYG@tc##1{\textcolor[rgb]{1.00,0.00,0.00}{##1}}}
\expandafter\def\csname PYG@tok@cm\endcsname{\let\PYG@it=\textit\def\PYG@tc##1{\textcolor[rgb]{0.25,0.50,0.56}{##1}}}
\expandafter\def\csname PYG@tok@vg\endcsname{\def\PYG@tc##1{\textcolor[rgb]{0.73,0.38,0.84}{##1}}}
\expandafter\def\csname PYG@tok@m\endcsname{\def\PYG@tc##1{\textcolor[rgb]{0.13,0.50,0.31}{##1}}}
\expandafter\def\csname PYG@tok@mh\endcsname{\def\PYG@tc##1{\textcolor[rgb]{0.13,0.50,0.31}{##1}}}
\expandafter\def\csname PYG@tok@cs\endcsname{\def\PYG@tc##1{\textcolor[rgb]{0.25,0.50,0.56}{##1}}\def\PYG@bc##1{\setlength{\fboxsep}{0pt}\colorbox[rgb]{1.00,0.94,0.94}{\strut ##1}}}
\expandafter\def\csname PYG@tok@ge\endcsname{\let\PYG@it=\textit}
\expandafter\def\csname PYG@tok@vc\endcsname{\def\PYG@tc##1{\textcolor[rgb]{0.73,0.38,0.84}{##1}}}
\expandafter\def\csname PYG@tok@il\endcsname{\def\PYG@tc##1{\textcolor[rgb]{0.13,0.50,0.31}{##1}}}
\expandafter\def\csname PYG@tok@go\endcsname{\def\PYG@tc##1{\textcolor[rgb]{0.20,0.20,0.20}{##1}}}
\expandafter\def\csname PYG@tok@cp\endcsname{\def\PYG@tc##1{\textcolor[rgb]{0.00,0.44,0.13}{##1}}}
\expandafter\def\csname PYG@tok@gi\endcsname{\def\PYG@tc##1{\textcolor[rgb]{0.00,0.63,0.00}{##1}}}
\expandafter\def\csname PYG@tok@gh\endcsname{\let\PYG@bf=\textbf\def\PYG@tc##1{\textcolor[rgb]{0.00,0.00,0.50}{##1}}}
\expandafter\def\csname PYG@tok@ni\endcsname{\let\PYG@bf=\textbf\def\PYG@tc##1{\textcolor[rgb]{0.84,0.33,0.22}{##1}}}
\expandafter\def\csname PYG@tok@nl\endcsname{\let\PYG@bf=\textbf\def\PYG@tc##1{\textcolor[rgb]{0.00,0.13,0.44}{##1}}}
\expandafter\def\csname PYG@tok@nn\endcsname{\let\PYG@bf=\textbf\def\PYG@tc##1{\textcolor[rgb]{0.05,0.52,0.71}{##1}}}
\expandafter\def\csname PYG@tok@no\endcsname{\def\PYG@tc##1{\textcolor[rgb]{0.38,0.68,0.84}{##1}}}
\expandafter\def\csname PYG@tok@na\endcsname{\def\PYG@tc##1{\textcolor[rgb]{0.25,0.44,0.63}{##1}}}
\expandafter\def\csname PYG@tok@nb\endcsname{\def\PYG@tc##1{\textcolor[rgb]{0.00,0.44,0.13}{##1}}}
\expandafter\def\csname PYG@tok@nc\endcsname{\let\PYG@bf=\textbf\def\PYG@tc##1{\textcolor[rgb]{0.05,0.52,0.71}{##1}}}
\expandafter\def\csname PYG@tok@nd\endcsname{\let\PYG@bf=\textbf\def\PYG@tc##1{\textcolor[rgb]{0.33,0.33,0.33}{##1}}}
\expandafter\def\csname PYG@tok@ne\endcsname{\def\PYG@tc##1{\textcolor[rgb]{0.00,0.44,0.13}{##1}}}
\expandafter\def\csname PYG@tok@nf\endcsname{\def\PYG@tc##1{\textcolor[rgb]{0.02,0.16,0.49}{##1}}}
\expandafter\def\csname PYG@tok@si\endcsname{\let\PYG@it=\textit\def\PYG@tc##1{\textcolor[rgb]{0.44,0.63,0.82}{##1}}}
\expandafter\def\csname PYG@tok@s2\endcsname{\def\PYG@tc##1{\textcolor[rgb]{0.25,0.44,0.63}{##1}}}
\expandafter\def\csname PYG@tok@vi\endcsname{\def\PYG@tc##1{\textcolor[rgb]{0.73,0.38,0.84}{##1}}}
\expandafter\def\csname PYG@tok@nt\endcsname{\let\PYG@bf=\textbf\def\PYG@tc##1{\textcolor[rgb]{0.02,0.16,0.45}{##1}}}
\expandafter\def\csname PYG@tok@nv\endcsname{\def\PYG@tc##1{\textcolor[rgb]{0.73,0.38,0.84}{##1}}}
\expandafter\def\csname PYG@tok@s1\endcsname{\def\PYG@tc##1{\textcolor[rgb]{0.25,0.44,0.63}{##1}}}
\expandafter\def\csname PYG@tok@gp\endcsname{\let\PYG@bf=\textbf\def\PYG@tc##1{\textcolor[rgb]{0.78,0.36,0.04}{##1}}}
\expandafter\def\csname PYG@tok@sh\endcsname{\def\PYG@tc##1{\textcolor[rgb]{0.25,0.44,0.63}{##1}}}
\expandafter\def\csname PYG@tok@ow\endcsname{\let\PYG@bf=\textbf\def\PYG@tc##1{\textcolor[rgb]{0.00,0.44,0.13}{##1}}}
\expandafter\def\csname PYG@tok@sx\endcsname{\def\PYG@tc##1{\textcolor[rgb]{0.78,0.36,0.04}{##1}}}
\expandafter\def\csname PYG@tok@bp\endcsname{\def\PYG@tc##1{\textcolor[rgb]{0.00,0.44,0.13}{##1}}}
\expandafter\def\csname PYG@tok@c1\endcsname{\let\PYG@it=\textit\def\PYG@tc##1{\textcolor[rgb]{0.25,0.50,0.56}{##1}}}
\expandafter\def\csname PYG@tok@kc\endcsname{\let\PYG@bf=\textbf\def\PYG@tc##1{\textcolor[rgb]{0.00,0.44,0.13}{##1}}}
\expandafter\def\csname PYG@tok@c\endcsname{\let\PYG@it=\textit\def\PYG@tc##1{\textcolor[rgb]{0.25,0.50,0.56}{##1}}}
\expandafter\def\csname PYG@tok@mf\endcsname{\def\PYG@tc##1{\textcolor[rgb]{0.13,0.50,0.31}{##1}}}
\expandafter\def\csname PYG@tok@err\endcsname{\def\PYG@bc##1{\setlength{\fboxsep}{0pt}\fcolorbox[rgb]{1.00,0.00,0.00}{1,1,1}{\strut ##1}}}
\expandafter\def\csname PYG@tok@kd\endcsname{\let\PYG@bf=\textbf\def\PYG@tc##1{\textcolor[rgb]{0.00,0.44,0.13}{##1}}}
\expandafter\def\csname PYG@tok@ss\endcsname{\def\PYG@tc##1{\textcolor[rgb]{0.32,0.47,0.09}{##1}}}
\expandafter\def\csname PYG@tok@sr\endcsname{\def\PYG@tc##1{\textcolor[rgb]{0.14,0.33,0.53}{##1}}}
\expandafter\def\csname PYG@tok@mo\endcsname{\def\PYG@tc##1{\textcolor[rgb]{0.13,0.50,0.31}{##1}}}
\expandafter\def\csname PYG@tok@mi\endcsname{\def\PYG@tc##1{\textcolor[rgb]{0.13,0.50,0.31}{##1}}}
\expandafter\def\csname PYG@tok@kn\endcsname{\let\PYG@bf=\textbf\def\PYG@tc##1{\textcolor[rgb]{0.00,0.44,0.13}{##1}}}
\expandafter\def\csname PYG@tok@o\endcsname{\def\PYG@tc##1{\textcolor[rgb]{0.40,0.40,0.40}{##1}}}
\expandafter\def\csname PYG@tok@kr\endcsname{\let\PYG@bf=\textbf\def\PYG@tc##1{\textcolor[rgb]{0.00,0.44,0.13}{##1}}}
\expandafter\def\csname PYG@tok@s\endcsname{\def\PYG@tc##1{\textcolor[rgb]{0.25,0.44,0.63}{##1}}}
\expandafter\def\csname PYG@tok@kp\endcsname{\def\PYG@tc##1{\textcolor[rgb]{0.00,0.44,0.13}{##1}}}
\expandafter\def\csname PYG@tok@w\endcsname{\def\PYG@tc##1{\textcolor[rgb]{0.73,0.73,0.73}{##1}}}
\expandafter\def\csname PYG@tok@kt\endcsname{\def\PYG@tc##1{\textcolor[rgb]{0.56,0.13,0.00}{##1}}}
\expandafter\def\csname PYG@tok@sc\endcsname{\def\PYG@tc##1{\textcolor[rgb]{0.25,0.44,0.63}{##1}}}
\expandafter\def\csname PYG@tok@sb\endcsname{\def\PYG@tc##1{\textcolor[rgb]{0.25,0.44,0.63}{##1}}}
\expandafter\def\csname PYG@tok@k\endcsname{\let\PYG@bf=\textbf\def\PYG@tc##1{\textcolor[rgb]{0.00,0.44,0.13}{##1}}}
\expandafter\def\csname PYG@tok@se\endcsname{\let\PYG@bf=\textbf\def\PYG@tc##1{\textcolor[rgb]{0.25,0.44,0.63}{##1}}}
\expandafter\def\csname PYG@tok@sd\endcsname{\let\PYG@it=\textit\def\PYG@tc##1{\textcolor[rgb]{0.25,0.44,0.63}{##1}}}

\def\PYGZbs{\char`\\}
\def\PYGZus{\char`\_}
\def\PYGZob{\char`\{}
\def\PYGZcb{\char`\}}
\def\PYGZca{\char`\^}
\def\PYGZam{\char`\&}
\def\PYGZlt{\char`\<}
\def\PYGZgt{\char`\>}
\def\PYGZsh{\char`\#}
\def\PYGZpc{\char`\%}
\def\PYGZdl{\char`\$}
\def\PYGZhy{\char`\-}
\def\PYGZsq{\char`\'}
\def\PYGZdq{\char`\"}
\def\PYGZti{\char`\~}
% for compatibility with earlier versions
\def\PYGZat{@}
\def\PYGZlb{[}
\def\PYGZrb{]}
\makeatother

\renewcommand\PYGZsq{\textquotesingle}

\begin{document}

\maketitle
\tableofcontents
\phantomsection\label{index::doc}


Contenu:


\chapter{AngularJS}
\label{angularjs:developper-avec-angularjs}\label{angularjs::doc}\label{angularjs:angularjs}

\section{Introduction}
\label{angularjs:introduction}
AngularJS est un framework Javascript crée et maintenu depuis 2009 par Google, l'entreprise a l'origine du populaire moteur de recherche, du client Gmail mais aussi dans notre cas de beaucoup d'outils pour les développeurs. En effet outre le fait de créer des outils de programmation comme AngularJS, Google propose des outils d'analyse (\href{http://google.com/analytics}{Analytics}), des outils de stockage (\href{https://cloud.google.com}{Google Cloud}) ou encore des serveurs pout hébérger des applications. Une entreprise non seulement très présente dans le domaine public mais également dans de le soutien et la recherche dans les technologies informatiques. Pour revenir à ce qui nous intéresse, ils ont donc développé le framework AngularJS. Un framework est un ensemble de structures et d'outils programmés dans un language et reutilisables permettant de faciliter la programmation d'applications. On trouve par exemple chez PHP le framework \href{http://symfony.com}{Symfony}, chez Ruby Ruby on Rails \textless{}\href{http://rubyonrails.com}{http://rubyonrails.com}\textgreater{}{}`\_, chez Python évidemment Django que nous utilisons pour notre site web et finalement chez Javascript AngularJS, mais encore comme concurrent \href{http://emberjs.com}{Ember} ou \href{http://backbonejs.org}{Backbone}.

AngularJS permet de développer une application web complète. C'est-à-dire qu'il offre plusieurs outils indispensables. D'abord un système de routes qui permet de lier des URLs avec des pages différentes, en clair on peut dire à notre application que quand l'utilisateur entre \emph{monsite.com/contact} dans son navigateur, il aille chercher et qu'il montre le fichier contact.html. Aussi la prise en charge de formulaire avec la possibilité de récupérer les informations entrées par l'utilisateur ou de faire une validation du formulaire, de vérifier les informations entrées. AngularJS ne sait pas comment communiquer directement avec une base de données, par contre il est tout à fait capable d'utliser les données d'une base de données fournies par un serveur intermédiaire. Cela permet donc de concevoir des sites web dynamiques. Nous éclaircisserons ce concept plus tard quand nous étudierons l'intégration d'Angular avec Django.


\section{Spécificités et avantages}
\label{angularjs:specificites-et-avantages}
Qu'est-ce qui fait qu'AngularJS puisse se démarquer de ses concurrents ? En bref qu'elles sont ses fonctionnalités qui facilitent tant la vie des développeurs ?


\subsection{Du HTML expressif}
\label{angularjs:du-html-expressif}
Le premier ``miracle'' du framework est de transformer le HTML statique en un language expressif et dynamique. A la base le HTML est un syntaxe qui permet de structurer une page internet à l'aide de balises qui définissent leur contenu comme par exemple les balises qui délimitent des paragraphes, les tires ou encore des formulaires, des images, des vidéos, etc. Et tradionnellement on utilise un autre language comme Python pour transformer la page et y insérer notre contenu dynamique qui provient de la base de données, concrétement il pourrait s'agir d'afficher une liste d'articles dans la page.


\chapter{Fonctionnalités}
\label{functionalities::doc}\label{functionalities:fonctionnalites}

\section{Les professeurs}
\label{functionalities:les-professeurs}
Les comptes professeur du site web ont la possibilité de pouvoir rédiger des cours complets qui seront ensuite consultable par les autres utlisateurs du site, principalement les élèves. L'objectif est de pouvoir offrir du contenu théorique d'apprentissage en complément à la partie pratiques, c'est-à-dire les exercices.


\subsection{Création}
\label{functionalities:creation}
La première étape du processeur de rédaction consiste à créer le cours en entrant les informations basiques du cours étant le nom, la description, la catégorie et la difficulté. Pour le nom et la description, ces champs parlent d'eux soi-même, il s'agit de rentrer des informations pertinentes afin que le cours puisse attirer les utlisateurs et puisse correspondre au contenu. Pour la catégorie, sur le site nous avons mis en place des thèmes qui contiennent eux-même des chapitres dans lequelles se placent des cours, par exemple un cours sur les tangentes qui se place dans le chapitre ``Les cercles'' qui lui-même est dans le thème ``Géométrie''. Ainsi les différentes resources du site sont classée et l'élève retrouve facilement la matière qui l'intéresse. Et finalement le professeur rentre la difficulté, sur une échelle de 1 à 3, de facile à difficile.

Une fois les champs dûment complétés, il suffit à l'auteur de cliquer sur le button ``Créer'' pour être rédiriger sur la page de rédaction où il pourra commencer à écrire.

Si l'auteur veut éditer les informations rentré lors de la création, il doit se rendre sur la page prévue à cette effet en cliquant sur le bouton ``Informations générales'' en haut de l'interface de rédaction.


\subsection{Rédaction}
\label{functionalities:redaction}

\subsubsection{Structurer son cours}
\label{functionalities:structurer-son-cours}
Pour structurer son cours, l'auteur peut tout d'abord créer plusieurs pages ayant chacune un titre. Ensuite dans chacune de ces pages, le contenu se découpe en plusieurs sections avec un titre et un contenu. L'avantage de cette structure est que le site génère automatiquement un sommaire interactif du cours en se basant sur le titre des pages et celui des sections.

Le fait de créer des pages ou des sections et de modifier le contenu se fait de manière claire, simple et surtout rapide. Le professeur travaille sans jamais devoir recharger sa page, action généralement lente. Pour ajouter une section, un button se trouve en bas de page et en cliquant dessus une nouelle zone d'édition va simplement apparaître. En bas de la page se trouve également d'une part le button pour ajouter une nouvelle page ainsi que la liste des différentes pages déjà crées. Il peut donc soit créer une page en cliquant ce qui a pour effet de d'afficher une nouvelle page vierge, ou alors de naviguer entre les pages du cours pour editer le contenu en cliquant sur le numéro des pages. Il se peut qu'on veuille réorganiser les sections ou les pages dans un ordre différent ou des les supprimer. Cela est tout à fait possible. Pour les sections de la page, à côté de chacune d'entre elle se trouve trois boutons: celui pour supprimer la section et les deux autres pour monter ou descendre une section. Lorsqu'on clique sur l'un des trois la page est mise à jour automatiquement et instantanément. En ce qui concerne les pages, il suffit de se rendre sur la page spéciale en cliquer sur ``Les pages'' en haut de la page pour retrouver le même système que pour les sections.

En ce qui concerne l'enregistrement, un se fait automatiquement toutes les minutes et si on veut quitter la page, un bouton en haut est prévu pour enregister le travail manuellement. De plus si on crée ou change de page, si on réorganise celles-ci ou les sections, le cours est aussi sauvegardé.


\subsubsection{Mise en forme}
\label{functionalities:mise-en-forme}
Pour mettre en forme le texte, le rédacteur utlise la syntaxe populaire Markdown. La syntaxe consiste à mettre des marques avec des symboles dans le texte qui seront ensuite interprétés. Par exemple pour mettre en italique on entoure le mot de ``*''. Ainsi il peut facilement mettre en gras, ajouter des images, faire des tableaux et des listes, etc. Si il n'est pas encore à l'aise avec la syntaxe, le professeur peut à tout moment cliquer sur le button d'aide, représenté par un point d'interogation, pour avoir un rapide aperçu de l'ensemble des fonctionnalités du Markdown.

De plus, comme les cours visant avant tout un contenu mathématique, l'auteur peut (et doit) baliser le contenu mathématique afin que celui-ci soit formaté pour le rendu final. La notation et l'affichage des formules mathématique est celui du populaire format Latex. Par conséquent, il suffit au professeur, dans son texte, de mettre les balises et ensuite d'écrire sa formule mathématique en utilisant la syntaxe. Il y a deux types de balises: ``(...)'' est utilisé pour intégrer des mathématiques directement dans le texte tandis que la balise ``\textbar{}\textbar{}'' est utlisé pour un affichage en block, c'est-à-dire sur une nouvelle ligne, séparée. A l'instar du Markdown, on peut se référer à la section d'aide pour se familiariser avec la syntaxe.

L'auteur n'ayant pas un aperçu direct du rendu de son texte, il peut donc cliquer sur le bouton en haut ``Aperçu'' pour regarder comment le contenu sera réellement. Il n'a pas besoin de sauvegarder avant de faire l'aperçu.


\subsubsection{Ajouter des médias}
\label{functionalities:ajouter-des-medias}
Evidemment l'auteur peut aisément inclure des vidéos et des images pour enrichir son contenu. Pour se faire, il y a d'abord la possibilité d'inclure une image provenant d'internet en utilisant la syntaxe Markdown et le lien de l'image, tandis que pour la vidéo on peut églalement inclure une de Youtube du même procédé que l'image (voir aide Markdown). L'autre possibilité est de directement téléverser l'image sur le site. Cliquer simplement sur l'icône vidéo ou image en haut de la page, une fenêtre pop-up s'ouvirira pour téléverser le fichier. Une fois l'upload terminé, le lien à inclure dans le texte est automatiquement généré et il suffit donc de le copier et de le coller à l'endroit désiré.


\subsubsection{Publier un cours}
\label{functionalities:publier-un-cours}
Par défaut, le cours créé par un professeur n'est pas visible par les utilisateurs du site afin de laisser le temps à l'auteur de rédiger l'entier du cours. Lorsque il estime que le cours est prêt à être lu, il peut tout simplement cliquer sur le bouton ``Publier'' tout en haut de la page et le cours sera visible. Il peut bien sûr à tout moment retirer son cours de la liste, le rendre non visionnable, s'il désire y apporter des modifications ou estime que son cours ne doit plus être sur le site.


\subsubsection{Editer son cours}
\label{functionalities:editer-son-cours}
Dans son dashboard, le professeur voit tous les cours qu'il a crées. S'il clique sur le bouton éditer d'un de ceux-là, il retournera sur l'interface d'édition de son cours pour pouvoir y apporter les modifications désirées.


\section{Les étudiants}
\label{functionalities:les-etudiants}
Tous les cours publiés sont donc consultable par les étudiants du site. Ils se rendent sur la section du site dédié aux cours et ont une liste du contenu disponible, dans lequelle ils peuvent faire une recherche ou trier par catégorie. Outre la possibilité de lire le cours et d'en feuilleter les pages, l'élève a quelques autres possibilités.


\subsection{Les favoris}
\label{functionalities:les-favoris}
S'il apprécie particulièrement un cours, le trouve utile ou veut le retrouver par la suite, l'étudiant peut l'ajouter en favoris comme les favoris d'un navigateur par exemple. Pour ce faire, il y a une étoile en haut de chaque page lorsqu'on lit un cours et qu'on est connecté au site. Elle est d'abord vide, ce qui signifie que le cours n'appartient pas aux favoris de la personne, et si on clique, l'étoile sera pleine et donc signifie que le cours est ajouté à sa liste. Il retrouvera ensuite dans son dashbord cette liste qui lui permet d'accéder rapidement à ses cours préférés. L'élève peut évidemment retiré un favoris de la même manière qu'il l'a ajouté.


\subsection{Les commentaires}
\label{functionalities:les-commentaires}
Une autre fonctionnalité est de pouvoir commenter un cours. En bas de chaque page du cours se trouve la section de commentaire qui rassemble tous les messages des utilisateurs. Il peut s'agir de remarques, compliments ou questions des lecteurs. Il y a en dessous des commentaires une zone de texte où l'on peut écrire son propre commentaire et ensuite le poster pour qu'il soit instantanément visible par les autres. Pour viser un contenu plus spécifique, il y a aussi la possibilité de commenter une section seule. En cliquant sur l'icône en forme de message en bas à gauche d'une section du cours une fenêtre s'ouvrira avec d'affiché seulement les commentaires de la section et dès lors le système est le même que pour les commentaires normaux.


\subsection{La progression}
\label{functionalities:la-progression}
L'élève peut aussi profiter d'un système de d'indication de progression. L'objectif est de pouvoir faciliter l'apprentissage de l'élève à travers le cours et de pouvoir l'aider à suivre les cours et à identifier les zones de faiblesse. A l'instar des commentaires, en bas de chaque page d'un cours se situent deux bouton intitulés ``Compris'' et ``A relire''. A la fin de sa lecture de la page, il est conseillé à l'élève de cliquer sur l'un des deux car cela lui permettra ensuite de se situer. D'une part le site se chargera de calculer le pourcentage du cours effectué en se basant sur les pages comprises et d'autre part le site lui proposera les pages qu'il faut relire. Toutes ces informations sont disponibles dans son dashboard.


\chapter{Modèle relationnel}
\label{models:modele-relationnel}\label{models::doc}

\chapter{RestLess}
\label{restless::doc}\label{restless:restless}

\chapter{Affichage des mathématiques}
\label{maths:affichage-des-mathematiques}\label{maths::doc}

\chapter{Guide du développeur}
\label{guide:guide-du-developpeur}\label{guide::doc}

\chapter{Indices et tables}
\label{index:indices-et-tables}\begin{itemize}
\item {} 
\emph{genindex}

\item {} 
\emph{modindex}

\item {} 
\emph{search}

\end{itemize}



\renewcommand{\indexname}{Index}
\printindex
\end{document}
